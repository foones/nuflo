
\usepackage{xspace}
\usepackage[all]{xy}
\usepackage{wrapfig}

\usepackage{amsmath}
\usepackage{amssymb}
\usepackage{mathrsfs}
\usepackage{xcolor}
\usepackage{mathtools}

\usepackage{hyperref}
\usepackage{chngcntr}
\usepackage{apptools}

\message{<Paul Taylor's Proof Trees, 2 August 1996>}
%% Build proof tree for Natural Deduction, Sequent Calculus, etc.
%% WITH SHORTENING OF PROOF RULES!
%% Paul Taylor, begun 10 Oct 1989
%% *** THIS IS ONLY A PRELIMINARY VERSION AND THINGS MAY CHANGE! ***
%%
%% 2 Aug 1996: fixed \mscount and \proofdotnumber
%%
%%      \prooftree
%%              hyp1            produces:
%%              hyp2
%%              hyp3            hyp1    hyp2    hyp3
%%      \justifies              -------------------- rulename
%%              concl                   concl
%%      \thickness=0.08em
%%      \shiftright 2em
%%      \using
%%              rulename
%%      \endprooftree
%%
%% where the hypotheses may be similar structures or just formulae.
%%
%% To get a vertical string of dots instead of the proof rule, do
%%
%%      \prooftree                      which produces:
%%              [hyp]
%%      \using                                  [hyp]
%%              name                              .
%%      \proofdotseparation=1.2ex                 .name
%%      \proofdotnumber=4                         .
%%      \leadsto                                  .
%%              concl                           concl
%%      \endprooftree
%%
%% Within a prooftree, \[ and \] may be used instead of \prooftree and
%% \endprooftree; this is not permitted at the outer level because it
%% conflicts with LaTeX. Also,
%%      \Justifies
%% produces a double line. In LaTeX you can use \begin{prooftree} and
%% \end{prootree} at the outer level (however this will not work for the inner
%% levels, but in any case why would you want to be so verbose?).
%%
%% All of of the keywords except \prooftree and \endprooftree are optional
%% and may appear in any order. They may also be combined in \newcommand's
%% eg "\def\Cut{\using\sf cut\thickness.08em\justifies}" with the abbreviation
%% "\prooftree hyp1 hyp2 \Cut \concl \endprooftree". This is recommended and
%% some standard abbreviations will be found at the end of this file.
%%
%% \thickness specifies the breadth of the rule in any units, although
%% font-relative units such as "ex" or "em" are preferable.
%% It may optionally be followed by "=".
%% \proofrulebreadth=.08em or \setlength\proofrulebreadth{.08em} may also be
%% used either in place of \thickness or globally; the default is 0.04em.
%% \proofdotseparation and \proofdotnumber control the size of the
%% string of dots
%%
%% If proof trees and formulae are mixed, some explicit spacing is needed,
%% but don't put anything to the left of the left-most (or the right of
%% the right-most) hypothesis, or put it in braces, because this will cause
%% the indentation to be lost.
%%
%% By default the conclusion is centered wrt the left-most and right-most
%% immediate hypotheses (not their proofs); \shiftright or \shiftleft moves
%% it relative to this position. (Not sure about this specification or how
%% it should affect spreading of proof tree.)
%
% global assignments to dimensions seem to have the effect of stretching
% diagrams horizontally.
%
%%==========================================================================

\def\introrule{{\cal I}}\def\elimrule{{\cal E}}%%
\def\andintro{\using{\land}\introrule\justifies}%%
\def\impelim{\using{\Rightarrow}\elimrule\justifies}%%
\def\allintro{\using{\forall}\introrule\justifies}%%
\def\allelim{\using{\forall}\elimrule\justifies}%%
\def\falseelim{\using{\bot}\elimrule\justifies}%%
\def\existsintro{\using{\exists}\introrule\justifies}%%

%% #1 is meant to be 1 or 2 for the first or second formula
\def\andelim#1{\using{\land}#1\elimrule\justifies}%%
\def\orintro#1{\using{\lor}#1\introrule\justifies}%%

%% #1 is meant to be a label corresponding to the discharged hypothesis/es
\def\impintro#1{\using{\Rightarrow}\introrule_{#1}\justifies}%%
\def\orelim#1{\using{\lor}\elimrule_{#1}\justifies}%%
\def\existselim#1{\using{\exists}\elimrule_{#1}\justifies}

%%==========================================================================

\newdimen\proofrulebreadth \proofrulebreadth=.05em
\newdimen\proofdotseparation \proofdotseparation=1.25ex
\newdimen\proofrulebaseline \proofrulebaseline=2ex
\newcount\proofdotnumber \proofdotnumber=3
\let\then\relax
\def\hfi{\hskip0pt plus.0001fil}
\mathchardef\squigto="3A3B
%
% flag where we are
\newif\ifinsideprooftree\insideprooftreefalse
\newif\ifonleftofproofrule\onleftofproofrulefalse
\newif\ifproofdots\proofdotsfalse
\newif\ifdoubleproof\doubleprooffalse
\let\wereinproofbit\relax
%
% dimensions and boxes of bits
\newdimen\shortenproofleft
\newdimen\shortenproofright
\newdimen\proofbelowshift
\newbox\proofabove
\newbox\proofbelow
\newbox\proofrulename
%
% miscellaneous commands for setting values
\def\shiftproofbelow{\let\next\relax\afterassignment\setshiftproofbelow\dimen0 }
\def\shiftproofbelowneg{\def\next{\multiply\dimen0 by-1 }%
\afterassignment\setshiftproofbelow\dimen0 }
\def\setshiftproofbelow{\next\proofbelowshift=\dimen0 }
\def\setproofrulebreadth{\proofrulebreadth}

%=============================================================================
\def\prooftree{% NESTED ZERO (\ifonleftofproofrule)
%
% first find out whether we're at the left-hand end of a proof rule
\ifnum  \lastpenalty=1
\then   \unpenalty
\else   \onleftofproofrulefalse
\fi
%
% some space on left (except if we're on left, and no infinity for outermost)
\ifonleftofproofrule
\else   \ifinsideprooftree
        \then   \hskip.5em plus1fil
        \fi
\fi
%
% begin our proof tree environment
\bgroup% NESTED ONE (\proofbelow, \proofrulename, \proofabove,
%               \shortenproofleft, \shortenproofright, \proofrulebreadth)
\setbox\proofbelow=\hbox{}\setbox\proofrulename=\hbox{}%
\let\justifies\proofover\let\leadsto\proofoverdots\let\Justifies\proofoverdbl
\let\using\proofusing\let\[\prooftree
\ifinsideprooftree\let\]\endprooftree\fi
\proofdotsfalse\doubleprooffalse
\let\thickness\setproofrulebreadth
\let\shiftright\shiftproofbelow \let\shift\shiftproofbelow
\let\shiftleft\shiftproofbelowneg
\let\ifwasinsideprooftree\ifinsideprooftree
\insideprooftreetrue
%
% now begin to set the top of the rule (definitions local to it)
\setbox\proofabove=\hbox\bgroup$\displaystyle % NESTED TWO
\let\wereinproofbit\prooftree
%
% these local variables will be copied out:
\shortenproofleft=0pt \shortenproofright=0pt \proofbelowshift=0pt
%
% flags to enable inner proof tree to detect if on left:
\onleftofproofruletrue\penalty1
}

%=============================================================================
% end whatever box and copy crucial values out of it
\def\eproofbit{% NESTED TWO
%
% various hacks applicable to hypothesis list 
\ifx    \wereinproofbit\prooftree
\then   \ifcase \lastpenalty
        \then   \shortenproofright=0pt  % 0: some other object, no indentation
        \or     \unpenalty\hfil         % 1: empty hypotheses, just glue
        \or     \unpenalty\unskip       % 2: just had a tree, remove glue
        \else   \shortenproofright=0pt  % eh?
        \fi
\fi
%
% pass out crucial values from scope
\global\dimen0=\shortenproofleft
\global\dimen1=\shortenproofright
\global\dimen2=\proofrulebreadth
\global\dimen3=\proofbelowshift
\global\dimen4=\proofdotseparation
\global\count255=\proofdotnumber
%
% end the box
$\egroup  % NESTED ONE
%
% restore the values
\shortenproofleft=\dimen0
\shortenproofright=\dimen1
\proofrulebreadth=\dimen2
\proofbelowshift=\dimen3
\proofdotseparation=\dimen4
\proofdotnumber=\count255
}

%=============================================================================
\def\proofover{% NESTED TWO
\eproofbit % NESTED ONE
\setbox\proofbelow=\hbox\bgroup % NESTED TWO
\let\wereinproofbit\proofover
$\displaystyle
}%
%
%=============================================================================
\def\proofoverdbl{% NESTED TWO
\eproofbit % NESTED ONE
\doubleprooftrue
\setbox\proofbelow=\hbox\bgroup % NESTED TWO
\let\wereinproofbit\proofoverdbl
$\displaystyle
}%
%
%=============================================================================
\def\proofoverdots{% NESTED TWO
\eproofbit % NESTED ONE
\proofdotstrue
\setbox\proofbelow=\hbox\bgroup % NESTED TWO
\let\wereinproofbit\proofoverdots
$\displaystyle
}%
%
%=============================================================================
\def\proofusing{% NESTED TWO
\eproofbit % NESTED ONE
\setbox\proofrulename=\hbox\bgroup % NESTED TWO
\let\wereinproofbit\proofusing
\kern0.3em$
}

%=============================================================================
\def\endprooftree{% NESTED TWO
\eproofbit % NESTED ONE
% \dimen0 =     length of proof rule
% \dimen1 =     indentation of conclusion wrt rule
% \dimen2 =     new \shortenproofleft, ie indentation of conclusion
% \dimen3 =     new \shortenproofright, ie
%                space on right of conclusion to end of tree
% \dimen4 =     space on right of conclusion below rule
  \dimen5 =0pt% spread of hypotheses
% \dimen6, \dimen7 = height & depth of rule
%
% length of rule needed by proof above
\dimen0=\wd\proofabove \advance\dimen0-\shortenproofleft
\advance\dimen0-\shortenproofright
%
% amount of spare space below
\dimen1=.5\dimen0 \advance\dimen1-.5\wd\proofbelow
\dimen4=\dimen1
\advance\dimen1\proofbelowshift \advance\dimen4-\proofbelowshift
%
% conclusion sticks out to left of immediate hypotheses
\ifdim  \dimen1<0pt
\then   \advance\shortenproofleft\dimen1
        \advance\dimen0-\dimen1
        \dimen1=0pt
%       now it sticks out to left of tree!
        \ifdim  \shortenproofleft<0pt
        \then   \setbox\proofabove=\hbox{%
                        \kern-\shortenproofleft\unhbox\proofabove}%
                \shortenproofleft=0pt
        \fi
\fi
%
% and to the right
\ifdim  \dimen4<0pt
\then   \advance\shortenproofright\dimen4
        \advance\dimen0-\dimen4
        \dimen4=0pt
\fi
%
% make sure enough space for label
\ifdim  \shortenproofright<\wd\proofrulename
\then   \shortenproofright=\wd\proofrulename
\fi
%
% calculate new indentations
\dimen2=\shortenproofleft \advance\dimen2 by\dimen1
\dimen3=\shortenproofright\advance\dimen3 by\dimen4
%
% make the rule or dots, with name attached
\ifproofdots
\then
        \dimen6=\shortenproofleft \advance\dimen6 .5\dimen0
        \setbox1=\vbox to\proofdotseparation{\vss\hbox{$\cdot$}\vss}%
        \setbox0=\hbox{%
                \advance\dimen6-.5\wd1
                \kern\dimen6
                $\vcenter to\proofdotnumber\proofdotseparation
                        {\leaders\box1\vfill}$%
                \unhbox\proofrulename}%
\else   \dimen6=\fontdimen22\the\textfont2 % height of maths axis
        \dimen7=\dimen6
        \advance\dimen6by.5\proofrulebreadth
        \advance\dimen7by-.5\proofrulebreadth
        \setbox0=\hbox{%
                \kern\shortenproofleft
                \ifdoubleproof
                \then   \hbox to\dimen0{%
                        $\mathsurround0pt\mathord=\mkern-6mu%
                        \cleaders\hbox{$\mkern-2mu=\mkern-2mu$}\hfill
                        \mkern-6mu\mathord=$}%
                \else   \vrule height\dimen6 depth-\dimen7 width\dimen0
                \fi
                \unhbox\proofrulename}%
        \ht0=\dimen6 \dp0=-\dimen7
\fi
%
% set up to centre outermost tree only
\let\doll\relax
\ifwasinsideprooftree
\then   \let\VBOX\vbox
\else   \ifmmode\else$\let\doll=$\fi
        \let\VBOX\vcenter
\fi
% this \vbox or \vcenter is the actual output:
\VBOX   {\baselineskip\proofrulebaseline \lineskip.2ex
        \expandafter\lineskiplimit\ifproofdots0ex\else-0.6ex\fi
        \hbox   spread\dimen5   {\hfi\unhbox\proofabove\hfi}%
        \hbox{\box0}%
        \hbox   {\kern\dimen2 \box\proofbelow}}\doll%
%
% pass new indentations out of scope
\global\dimen2=\dimen2
\global\dimen3=\dimen3
\egroup % NESTED ZERO
\ifonleftofproofrule
\then   \shortenproofleft=\dimen2
\fi
\shortenproofright=\dimen3
%
% some space on right and flag we've just made a tree
\onleftofproofrulefalse
\ifinsideprooftree
\then   \hskip.5em plus 1fil \penalty2
\fi
}

%==========================================================================
% IDEAS
% 1.    Specification of \shiftright and how to spread trees.
% 2.    Spacing command \m which causes 1em+1fil spacing, over-riding
%       exisiting space on sides of trees and not affecting the
%       detection of being on the left or right.
% 3.    Hack using \@currenvir to detect LaTeX environment; have to
%       use \aftergroup to pass \shortenproofleft/right out.
% 4.    (Pie in the sky) detect how much trees can be "tucked in"
% 5.    Discharged hypotheses (diagonal lines).

\newcommand{\indrulename}[1]{\texttt{\textup{#1}}}
\newcommand{\indrule}[3]{
\ensuremath{\begin{array}{c}
  \prooftree #2
    \justifies #3
    \thickness=0.05em
    \using \indrulename{\scriptsize{#1}}
  \endprooftree
\end{array}}}

\renewcommand{\theenumi}{\arabic{enumi}}
\renewcommand{\theenumii}{\arabic{enumii}}
\renewcommand{\theenumiii}{\arabic{enumiii}}
\renewcommand{\theenumiv}{\arabic{enumiv}}
%
\renewcommand{\labelenumi}{\arabic{enumi}.}
\renewcommand{\labelenumii}{\arabic{enumi}.\arabic{enumii}}
\renewcommand{\labelenumiii}{\arabic{enumi}.\arabic{enumii}.\arabic{enumiii}}
\renewcommand{\labelenumiv}{\arabic{enumi}.\arabic{enumii}.\arabic{enumiii}.\arabic{enumiv}}
%
\makeatletter
\renewcommand\p@enumii{\theenumi.}
\renewcommand\p@enumiii{\theenumi.\theenumii.}
\renewcommand\p@enumiv{\theenumi.\theenumii.\theenumiii.}
\makeatother

%%% Theorem environments

\theoremstyle{break}
\newtheorem{dummythm}{dummythm}
\newtheorem{lemma}[dummythm]{Lema}
\newtheorem{convention}[dummythm]{Convención}
\newtheorem{proposition}[dummythm]{Proposición}
\newtheorem{theorem}[dummythm]{Teorema}
\newtheorem{corollary}[dummythm]{Corolario}

\theoremstyle{definition}
\newtheorem{definition}[dummythm]{Definición}
\newtheorem{example}[dummythm]{Ejemplo}
\newtheorem{exercise}[dummythm]{Ejercicio}
\newtheorem{algorithm}[dummythm]{Algoritmo}

\theoremstyle{remark}
\newtheorem{remark}[dummythm]{Observación}
\newtheorem{notation}[dummythm]{Notación}

\newcommand{\llem}[1]{\label{lemma:#1}}
\newcommand{\rlem}[1]{Lema~\ref{lemma:#1}}
\newcommand{\ldef}[1]{\label{def:#1}}
\newcommand{\rdef}[1]{Definición~\ref{def:#1}}
\newcommand{\lprop}[1]{\label{prop:#1}}
\newcommand{\rprop}[1]{Proposición~\ref{prop:#1}}
\newcommand{\lthm}[1]{\label{thm:#1}}
\newcommand{\rthm}[1]{Teorema~\ref{thm:#1}}
\newcommand{\lremark}[1]{\label{remark:#1}}
\newcommand{\rremark}[1]{Observación~\ref{remark:#1}}
\newcommand{\lcoro}[1]{\label{coro:#1}}
\newcommand{\rcoro}[1]{Corolario~\ref{coro:#1}}
\newcommand{\lsec}[1]{\label{section:#1}}
\newcommand{\rsec}[1]{Sección~\ref{section:#1}}
\newcommand{\lexample}[1]{\label{example:#1}}
\newcommand{\rexample}[1]{Ejemplo~\ref{example:#1}}
\newcommand{\leqn}[1]{\label{eqn:#1}}
\newcommand{\reqn}[1]{(\ref{eqn:#1})}
\newcommand{\lfig}[1]{\label{fig:#1}}
\newcommand{\rfig}[1]{Figura~\ref{fig:#1}}

%
\newcommand{\defn}[1]{{\bf #1}}
\newcommand{\eg}{{\em e.g.}\xspace}
\newcommand{\ie}{{\em i.e.}\xspace}
\newcommand{\ih}{{\em h.i.}\xspace}
\newcommand{\etal}{et al.\xspace}
\newcommand{\cf}{{\em cf.}\xspace}
\newcommand{\ST}{\ |\ }
\newcommand{\HS}{\hspace{.5cm}}
\newcommand{\HStight}{\hspace{.2cm}}
\newcommand{\VS}{\vspace{.5cm}}
\renewcommand{\emptyset}{\varnothing}
\newcommand{\set}[1]{\{#1\}}
\newcommand{\Nat}{\mathtt{Nat}}
\newcommand{\NN}{\mathbb{N}}
\newcommand{\eqdef}{\overset{\mathrm{def}}{=}}
\newcommand{\iffdef}{\overset{\mathrm{def}}{\iff}}
\newcommand{\eqih}{\overset{\mathrm{h.i.}}{=}}
\newcommand{\TODO}[1]{\textcolor{red}{TODO: #1}}
\newcommand{\FIT}[1]{\resizebox{\hsize}{!}{#1}}

\newcommand{\esub}[2]{[#1\backslash#2]}
\newcommand{\hetofull}[1]{\xmapsto{\rulename{#1}}}
\newcommand{\Var}{\mathsf{Var}}
\newcommand{\Cons}{\mathsf{Con}}
\newcommand{\Val}{\mathsf{Val}}
\newcommand{\Type}{\mathsf{Type}}

\newcommand{\Loc}{\mathsf{Loc}}
\newcommand{\Asg}{\mathsf{Env}}

%%%%
\newcommand{\var}{x}
\newcommand{\vartwo}{y}
\newcommand{\varthree}{z}
\newcommand{\varfour}{w}

\newcommand{\cons}{{\bf c}}
\newcommand{\constwo}{{\bf d}}
\newcommand{\consthree}{{\bf e}}

\newcommand{\tm}{t}
\newcommand{\tmtwo}{s}
\newcommand{\tmthree}{u}
\newcommand{\tmfour}{r}
\newcommand{\tmfive}{p}
\newcommand{\tmsix}{q}

\newcommand{\fix}[2]{\mathtt{fix}(#1.#2)}
\newcommand{\unit}{{\bf ok}}
\newcommand{\fail}{\mathtt{fail}}
\newcommand{\FAIL}{\bot}
\newcommand{\unif}{\overset{\bullet}{=}}
\newcommand{\sym}{\texttt{\textup{<\!?\!>}}}
\newcommand{\seq}{;}
\newcommand{\alt}{\oplus}
\newcommand{\assoc}[2]{[#1\backslash#2]}
\newcommand{\anon}{\mathtt{X}}
\newcommand{\anontwo}{\mathtt{Y}}

\newcommand{\lam}[2]{\lambda #1.\,#2}
\newcommand{\laml}[3]{\lambda^{#1} #2.\,#3}
\newcommand{\fresh}[2]{\nu #1.\,#2}
\newcommand{\sub}[2]{\{#1:=#2\}} % revisar sintaxis

\DeclareMathOperator{\supp}{supp}
\newcommand{\emptyEnv}{\epsilon}

\newcommand{\emptylist}{\epsilon}

\newcommand{\val}{\mathtt{v}}
\newcommand{\valtwo}{\mathtt{w}}
\newcommand{\aval}{\mathtt{a}}
\newcommand{\avaltwo}{\mathtt{b}}

\newcommand{\goals}{\mathsf{G}}
\newcommand{\goalstwo}{\mathsf{H}}

\newcommand{\fv}[1]{\mathsf{fv}(#1)}
\newcommand{\locs}[1]{\mathsf{locs}(#1)}

% Small-step semantics
\newcommand{\ctxof}[1]{\langle#1\rangle}
\newcommand{\ctxhole}{\Box}
\newcommand{\SEP}{\mid\!\mid}
\newcommand{\gctx}{\mathsf{C}}
\newcommand{\gctxof}[1]{\gctx\ctxof{#1}}
\newcommand{\wctx}{\mathsf{W}}
\newcommand{\wctxof}[1]{\wctx\ctxof{#1}}

\newcommand{\LeftRightarrow}{\Lleftarrow\!\!\!\!\Rrightarrow}
\newcommand{\structeq}{\equiv}
\newcommand{\permeq}{\sim}

% Normal forms
\newcommand{\rulename}[1]{\texttt{\textup{#1}}}

\newcommand{\topartial}{\rightharpoonup}
\newcommand{\tounifa}[1]{\mathrel{\rightsquigarrow_{\rulename{#1}}}}
%\newcommand{\toca}[1]{\xrightarrow{#1}}
%\newcommand{\rtoca}[1]{\mathrel{\xrightarrow{\rulename{#1}}\!\!\!\!\!\xrightarrow{}}}
%\newcommand{\topar}[1]{\xRightarrow{#1\,}}
\newcommand{\toca}[1]{
  \def\tmp{#1}\ifx\tmp\empty
    \rightarrow%
  \else
    \xrightarrow{\rulename{#1}}%
  \fi}
\newcommand{\rtoca}[1]{
  \def\tmp{#1}\ifx\tmp\empty
    \twoheadrightarrow%
  \else
    \mathrel{\xrightarrow{\rulename{#1}}\!\!\!\!\!\xrightarrow{}}%
  \fi}
\newcommand{\topar}[1]{
  \def\tmp{#1}\ifx\tmp\empty
    \Rightarrow
  \else
    \xRightarrow{#1\,}%
  \fi
}

\newcommand{\prog}{P}
\newcommand{\progtwo}{Q}
\newcommand{\progthree}{R}

% Locations
\newcommand{\loc}{\ell}
\newcommand{\loctwo}{\loc'}
\newcommand{\locthree}{\loc''}

\newcommand{\sctx}{\mathtt{L}}

% Substitutions and unification
\newcommand{\SUB}[1]{{}^{#1}}
\newcommand{\subst}{\sigma}
\newcommand{\substtwo}{\rho}
\newcommand{\substthree}{\tau}
\newcommand{\substa}{\alpha}
\newcommand{\scomp}{\cdot}
\newcommand{\sleq}{\lesssim}
\newcommand{\mgu}[1]{\mathsf{mgu}(#1)}
\newcommand{\bigalt}{\bigoplus}

\newcommand{\devterm}[1]{#1^\circ}
\newcommand{\devprog}[1]{#1^\bullet}
\newcommand{\devgoals}[1]{\mathsf{goals}(#1)}

\newcommand{\nprog}{\prog^\star}
\newcommand{\nprogtwo}{\progtwo^\star}
\newcommand{\ntm}{\tm^\star}
\newcommand{\ntmtwo}{\tmtwo^\star}
\newcommand{\stm}{S}
\newcommand{\stmtwo}{S'}
\newcommand{\stuck}{\bigtriangledown}

%%%% Types

\newcommand{\btyp}{\alpha}
\newcommand{\btyptwo}{\beta}
\newcommand{\btypthree}{\gamma}

\newcommand{\constyp}[1]{\mathcal{T}_{#1}}

\newcommand{\typ}{A}
\newcommand{\typtwo}{B}
\newcommand{\typthree}{C}

\newcommand{\emptyctx}{\emptyset}
\newcommand{\tctx}{\Gamma}
\newcommand{\tctxtwo}{\Delta}
\newcommand{\tctxthree}{\Sigma}
\newcommand{\fctx}{\Phi}
\newcommand{\fctxtwo}{\Psi}

%%%% Denotational semantics
\newcommand{\sembare}[1]{[\![#1]\!]}
\newcommand{\semtyp}[1]{[\![#1]\!]}
\newcommand{\seme}[2]{[\![#1]\!]_{#2}}
\newcommand{\semf}[3][\fctx]{[\![#2]\!]^{#1}_{#3}}
\newcommand{\semm}[3]{[\![#1]\!]\,#2\,#3}
\newcommand{\powerset}[1]{\mathcal{P}(#1)}
\newcommand{\binterp}[1]{\mathsf{S}_{#1}}
\newcommand{\cinterp}[1]{\underline{#1}}
\newcommand{\asg}{\rho}
\newcommand{\asgextend}[2]{[#1\mapsto#2]}

\newcommand{\metalam}[2]{\leftthreetimes #1.\,#2}

\newcommand{\obj}{a}
\newcommand{\objtwo}{b}
\newcommand{\objthree}{c}
\newcommand{\objfour}{d}
\newcommand{\objfun}{f}
\newcommand{\objfuntwo}{g}

\newcommand{\mem}{\mu}
\newcommand{\altt}{\boxplus}
\newcommand{\failt}{\textsc{fail}}

\newcommand{\lambdaunif}{\lambda^{\mathtt{U}}}

\newcommand{\codesym}{\mathtt{C}}
\newcommand{\allocsym}{\mathtt{A}}

\colorlet{darkgreen}{green!60!black}
\newcommand{\SeeAppendix}{\textcolor{darkgreen}{$\clubsuit$}}
\newcommand{\SeeAppendixRef}[1]{\textcolor{darkgreen}{$\clubsuit$\,#1}}
\newcommand{\WithProofs}[1]{\textcolor{darkgreen}{#1}}

% Example
\newcommand{\pairing}{\mathbf{t}}
\newcommand{\walg}[1]{\mathbb{W}[#1]}

% Language

\newcommand{\chr}[1]{\texttt{'}\textcolor{black}{\texttt{#1}}\texttt{'}}
\newcommand{\str}[1]{\texttt{"}\textcolor{black}{\texttt{#1}}\texttt{"}}
\newcommand{\token}[1]{\textcolor{darkgreen}{\texttt{#1}}}
\newcommand{\nonterminal}[1]{\textcolor{black}{{\it$\langle$#1$\rangle$}}}
\newcommand{\nonEmpty}[1]{#1$_{1}$}
\newcommand{\production}[2]{
  \noindent
  \begin{tabular}{lrp{10cm}}
  #1 & $\xrightarrow{\hspace{.5cm}}$ & #2
  \end{tabular}\\
}
\newcommand{\EMPTY}{$\epsilon$}
\newcommand{\ALT}{
  \\ & $\mid$ &
}

\newcommand{\nuflo}{Ñuflo\xspace}
\newcommand{\modulo}[1]{\textcolor{purple}{#1}}


